\section{Opisna statistika}

Za prikaz opisne statistike naših podatkov uporabimo R funkcijo \emph{summary}, ki nam vrne minimum, maksimum,
prvi in tretji kvartil in mediano.
Prav tako uporabimo še štiri druge ukaze:

\begin{itemize}
    \item \verb|(mx<-mean(mozgani$telteza))|\label{en:mx} nam prikaže \emph{vzorčno poprečje} telesne teže sesalcev,
    \item \verb|(sx<-sd(mozgani$telteza))|\label{en:sx} nam prikaže \emph{vzočni standardni odklon} telesne teže sesalcev.
    \item \verb|(my<-mean(mozgani$mozteza))|\label{en:my} nam prikaže \emph{vzorčno poprečje} možganske teže sesalcev,
    \item \verb|(sy<-sd(mozgani$mozteza))|\label{en:sy} nam prikaže \emph{vzočni standardni odklon} možganske teže sesalcev.
\end{itemize}

\noindent
Rezultati ukazov so prikazani spodaj:

\begin{table}[h]
    \centering
    \begin{tabular}{|c|c|c|c|c|c|}
    \hline
    \textbf{Min.} & \textbf{1. Kvartil} & \textbf{Mediana} & \textbf{Povprečje} & \textbf{3. Kvartil} & \textbf{Max.} \\
    0.005 & 0.768 & 3.385 & 208.840 & 53.830 & 6654.180 \\ \hline
    \multicolumn{3}{|c|}{\textbf{Vzor. Povpr.}} & \multicolumn{3}{c|}{\textbf{Vzor. Stand. Odkl.}} \\
    \multicolumn{3}{|c|}{208.8403} & \multicolumn{3}{c|}{920.9927} \\ \hline
    \end{tabular}
    \caption{Opisna statistika za telesno težo sesalcev}
    \label{tab:telteza}
    \end{table}

\begin{table}[h]
    \centering
    \begin{tabular}{|c|c|c|c|c|c|}
    \hline
    \textbf{Min.} & \textbf{1. Kvartil} & \textbf{Mediana} & \textbf{Povprečje} & \textbf{3. Kvartil} & \textbf{Max.} \\
    0.14 & 5.60 & 21.00 & 297.41 & 172.00 & 5711.86 \\ \hline
    \multicolumn{3}{|c|}{\textbf{Vzor. Povpr.}} & \multicolumn{3}{c|}{\textbf{Vzor. Stand. Odkl.}} \\
    \multicolumn{3}{|c|}{297.4095} & \multicolumn{3}{c|}{951.7869} \\ \hline
    \end{tabular}
    \caption{Opisna statistika za možgansko težo sesalcev}
    \label{tab:mozteza}
    \end{table}

Tabela \ref{tab:telteza} prikazuje opisno statistiko za telesno težo sesalcev, kjer opazimo, da telesna teža vzorca
variira med \verb|5 g| in \verb|6654 kg|, s povprečjem \verb|208.8 kg| in standardnim odklonom \verb|921 kg|.

Tabela \ref{tab:mozteza} prikazuje opisno statistiko za možgansko težo sesalcev, kjer opazimo, da možganska teža vzorca
variira med \verb|0.14 g| in \verb|5712 g| (5.7 kg), s povprečjem \verb|297.4 g| in standardnim odklonom \verb|951.79 g|.

Masi telesne in možganske teže nam pomagata pri izbiri mej na oseh razsevnega diagrama.