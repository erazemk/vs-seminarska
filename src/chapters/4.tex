\section{Formiranje linearnega regresijskega modela}

Linearni regresijski model lahko formiramo s pomočjo funkcije \verb|lm()|.
Ker moramo podatke prvo transformirati lahko vmesne rezultate shranimo v spremenljivke:

\begin{itemize}
    \item \verb|lgmteza <- log(mozgani$mozteza)| in
    \item \verb|lgtteza <- log(mozgani$telteza)|.
\end{itemize}

\noindent
Nato uporabimo ukaz \verb|(model <- lm(lgmteza~lgtteza, data = mozgani))|, ki nam vrne regresijsko premico
$\widehat{y} = 2.1781 + 0.7468$, oz. oceni odseka ($\widehat{a} = 2.1781$) in naklona ($\widehat{b} = 0.7468$).\\

\noindent
Oceni odseka in naklona sta izračunani s formulama \verb|a <- my - b * mx|, ki nam vrne rezultat \verb|[1] 2.1781|,
in \verb|b <- r * sy / sx|, ki nam vrne rezultat \verb|[1] 0.7468|, pri čemer smo \emph{r}, \emph{mx}, \emph{my},
\emph{sx} in \emph{sy} izračunali že prej.

\subsection{Točke visokega vzvoda in osamelci}

Identificirajmo točke visokega vzvoda in osamelce.
Vrednost \emph{x} je točka visokega vzvoda, če je njen vzvod večji od $\frac{4}{n}$.
Točke visokega vzvoda lahko izračunamo z ukazom

\verb|mozgani[hatvalues(model) > 4/57,]|, ki nam vrne naslednji rezultat

\begin{verbatim}
        vrsta               slovime                telteza   mozteza
    3   Blarina brevicauda  Rovka                  0.005     0.14
    16  Elephas maximus     Azijski slon           2547.070  4603.17
    30  Loxodonta africana  Afriski slon           6654.180  5711.86
    36  Myotis lucifugus    Majhni rjavi netopir   0.010     0.25
\end{verbatim}

Vrnjen rezultat nam pove, da imamo štiri točke visokega vzvoda, dve vrsti sesalcev imata zelo nizko telesno težo,
dve pa zelo visoko telesno težo.\\

Za podatke majhne in srednje velikosti vzorca je osamelec podatkovna točka, kateri ustreza standardizirani ostanek
izven intervala $[-2, 2]$.
Ostanke standardiziramo, ko jih delimo z njihovim standardnim odklonom, kar storimo z ukazom

\verb|mozgani[abs(rstandard(model)) > 2,]|, ki nam vrne naslednji rezultat

\begin{verbatim}
        vrsta                 slovime        telteza  mozteza
    29  Homo sapiens sapiens  Clovek         61.998   1320.020
    31  Macaca mulatta        Rezus makaki   6.800    179.003
\end{verbatim}

Dve podatkovni točki sta osamelca, oz. imata nenavadno nizko ali visoko težo možganov glede na telesno težo v
primerjavi z ostalimi podatki.