\section{Testiranje linearnosti modela in koeficient determinacije}

Poročilo o modelu lahko dobimo z ukazom \verb|summary(model)|, ki nam vrne spodnji (skrajšan) rezultat:

\begin{verbatim}
    Coefficients:
                Estimate Std. Error t value Pr(>|t|)    
    (Intercept)  2.17812    0.09487   22.96   <2e-16 ***
    lgtteza      0.74679    0.02773   26.93   <2e-16 ***
    ---
    Residual standard error: 0.6579 on 57 degrees of freedom
    Multiple R-squared:  0.9271,    Adjusted R-squared:  0.9259
    F-statistic: 725.4 on 1 and 57 DF,  p-value: < 2.2e-16
\end{verbatim}

Zanima nas samo testna statistika za testiranje linearnosti modela $T = 26.93$, z $df = 57$ prostorskimi
stopnjami, in p-vrednostjo $p = 2.2 \cdot 10^{16}$, ki je manjša od dane stopnje značilnosti 0.05.
S formalnim statističnim testiranjem smo potrdili, da linearni model ustreza podatkom.
Standardni odklon napak ocenjen s $S = 0.6579$, koeficient determinacije pa je enak $R^{2} = 0.9271$, kar je kvadrat
vzorčnega koeficienta korelacije. To pomeni, da 93\% variabilnosti telesne teže sesalcev pojasnjuje linearni regresijski
model.