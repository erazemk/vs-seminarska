\section{Interval predikcije za vrednost Y pri izbrani vrednosti X}

Pri predvidevanju velikosti možganov sesalcev nas zanima bodoča vrednost spremenljivke Y (teža možanov)
pri neki izbrani vrednosti spremenljivke X = $x_0$ (telesna teža).
Poleg predvidevane vrednosti $\widehat{y} = 2.1781 + 0.7468x_0$ za neke izbrane telesne teže sesalcev $x_0$ nas
zanimata tudi spodnja in zgornja meja, med katerima se nahaja velikost možganov sesalcev teh telesnih tež.
\emph{Interval predikcije} najdemo s pomočjo R funkcije \emph{predict()} in sicer z ukazoma
\verb|xlgteza = data.frame(lgtteza = log(c(100, 500, 2000)))| in \verb|exp(predict(model, xlgteza, interval = "predict"))|,
kar nam vrne rezultat:

\begin{table}[h]
    \centering
    \begin{tabular}{cccc}
    \textbf{}  & \textbf{fir} & \textbf{lwr} & \textbf{upr} \\
    \textbf{1} & 275.1184     & 72.05234     & 1050.489     \\
    \textbf{2} & 915.1696     & 236.22735    & 3545.463     \\
    \textbf{3} & 2576.9873    & 653.93940    & 10155.167   
    \end{tabular}
    \caption{Interval predikcije}
    \label{tab:interval-predikcije}
    \end{table}

\noindent
Pri tem prvi ukaz spremenljivki \emph{lgtteza} dodeli prosto izbrane parametre 100, 500 in 2000, ki predstavljajo
telesne teže sesalcev v kilogramih, za katere želimo izračunati predvidevano težo možganov, drugi ukaz pa
izračuna pričakovane vrednosti velikosti možganov, pri čimer uporablja že večkrat omenjeni linearni regresijski model,
imenovan \emph{model}, polje vrednosti \emph{xlgteza} iz prejšnjega ukaza in možnost \emph{interval = "predict"}.
Predikcijo vrednosti slučajne spremenljivke Y je smiselno narediti v okviru razpona \emph{x} vrednosti v vzorcu.

Iz tabele \ref{tab:interval-predikcije} razberemo, da so predvidene vrednosti teže možganov sesalcev s telesnimi masami
naslednje:

\begin{itemize}
    \item 100 kg je 275.12 g, s 95\% intervalom predikcije teže možganov [72.05, 1050.49],
    \item 500 kg je 915.17 g, s 95\% intervalom predikcije teže možganov [236.23, 3545.46],
    \item 2000 kg je 2576.99 g, s 95\% intervalom predikcije teže možganov [653.94, 10155.17].
\end{itemize}